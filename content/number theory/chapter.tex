% This is the LaTeX file for the chapter "Number theory" in KACTL (KTH ACM Contest Template Library)
% Written by Håkan Terelius, 2009-02-10

\chapter{Number theory}

	\section{Power modulo}
		\kactlimport{mod_power.h}

	\section{Primality}
		\kactlimport{eratosthenes.h}
		\kactlimport{prime_sieve.h}
		\kactlimport{miller_rabin.h}

	\section{Divisibility}
		\kactlimport{euclid.h}
		\kactlimport{phiFunction.h}
		\subsection{Bézout's identity}
		For $a \neq $, $b \neq 0$, then $d=gcd(a,b)$ is the smallest positive integer for which there are integer solutions to
		$$ax+by=d$$
		If $(x,y)$ is one solution, then all solutions are given by
		$$\left(x+\frac{kb}{\gcd(a,b)}, y-\frac{ka}{\gcd(a,b)}\right), \quad k\in\mathbb{Z}$$
		
		\import{divisibilityRules}
		

	\section{Chinese reminder theorem}
		\kactlimport{chinese.h}

	\section{Modular arithmetic}
		\kactlimport{modularArithmetic.h}

	\section{Primes}
		$p=962592769$ is such that $2^{21} \mid p-1$, which may be useful. For hashing
		use 970592641 (31-bit number), 31443539979727 (45-bit), 3006703054056749
		(52-bit). There are 78498 primes less than 1\,000\,000.

	\section{Estimates}
		$\sum_{d|n}d(d) = O(n \log n)$, where $d(n)$ is the number of divisors of $n$.
		$d(n)$ is at most around 100 for $n < 5e4$, 500 for $n < 1e7$ and 2000 for $n < 1e10$.

	\section{Factorisation}
		\kactlimport{factor.h}
