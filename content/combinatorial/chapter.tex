\chapter{Combinatorial}

\section{The Twelvefold Way}
	\import{twelvefoldway.tex}

\section{Permutations}
	\subsection{Factorial}
		\import{factorial.tex}
		\kactlimport{intperm.h}

	\subsection{Cycles}
		Let the number of $n$-permutations whose cycle lengths all belong to the set $S$ be denoted by $g_S(n)$. Then
		$$\sum_{n=0} ^\infty g_S(n) \frac{x^n}{n!} = \exp\left(\sum_{n\in S} \frac{x^n} {n} \right)$$

	\subsection{Derangements}
		Permutations of a set such that none of the elements appear in their original position.
		$$D(n) = (n-1)(D(n-1)+D(n-2)) = n D(n-1)+(-1)^n = \left\lfloor\frac{n!}{e}\right\rceil$$
		\kactlimport{derangements.h}

	\subsection{Involutions}
		An involution is a permutation with maximum cycle length 2, and it is its own inverse.
		$$a(n) = a(n-1) + (n-1)a(n-2)$$
		$$a(0) = a(1) = 1$$
		1, 1, 2, 4, 10, 26, 76, 232, 764, 2620, 9496, 35696, 140152

	\subsection{Stirling numbers of the first kind}
		$$s(n,k) = (-1)^{n-k}c(n,k)$$
		$c(n,k)$ is the unsigned Stirling numbers of the first kind, and they count the number of permutations on $n$ items with $k$ cycles.
		$$s(n,k) = s(n-1,k-1) - (n-1) s(n-1,k)$$
		$$s(0,0) = 1, s(n,0) = s(0,n) = 0$$
		$$c(n,k) = c(n-1,k-1) + (n-1) c(n-1,k)$$
		$$c(0,0) = 1, c(n,0)=c(0,n)=0$$

	\subsection{Eulerian numbers}
		Number of permutations $\pi \in S_n$ in which exactly $k$ elements are greater than the previous element. $k$ $j$:s s.t. $\pi(j)>\pi(j+1)$, $k+1$ $j$:s s.t. $\pi(j)\geq j$, $k$ $j$:s s.t. $\pi(j)>j$.
		$$E(n,k) = (n-k)E(n-1,k-1) + (k+1)E(n-1,k)$$
		$$E(n,0) = E(n,n-1) = 1$$
		$$E(n,k) = \sum_{j=0}^k(-1)^j\binom{n+1}{j}(k+1-j)^n$$

	\subsection{Burnside's lemma}
		Given a group $G$ of symmetries and a set $X$, the number of elements of $X$ \emph{up to symmetry} equals
		 \[ {\frac {1}{|G|}}\sum _{{g\in G}}|X^{g}|, \]
		 where $X^{g}$ are the elements fixed by $g$ ($g.x = x$).

		 If $f(n)$ counts "configurations" (of some sort) of length $n$, we can ignore rotational symmetry using $G = \mathbb Z_n$ to get
		 \[ g(n) = \frac 1 n \sum_{k=0}^{n-1}{f(\text{gcd}(n, k))} = \frac 1 n \sum_{k|n}{f(k)\phi(n/k)}. \]

\section{Partitions and subsets}
	\subsection{Partition function}
		Partitions of $n$ with exactly $k$ parts, $p(n,k)$, i.e., writing $n$ as a sum of $k$ positive integers, disregarding the order of the summands.
		$$p(n,k) = p(n-1,k-1)+p(n-k,k)$$
		$$p(0,0)=p(1,n)=p(n,n)=p(n,n-1)=1$$

		For partitions with any number of parts, $p(n)$ obeys
		\[ p(0) = 1,\ p(n) = \sum_{k \in \mathbb Z \setminus \{0\}}{(-1)^{k+1} p(n - k(3k-1) / 2)} \]
		\[ p(n) \sim 0.145 / n \cdot \exp(2.56 \sqrt{n}) \]

		\begin{center}
		\begin{tabular}{c|c@{\ }c@{\ }c@{\ }c@{\ }c@{\ }c@{\ }c@{\ }c@{\ }c@{\ }c@{\ }c@{\ }c@{\ }c}
			$n$    & 0 & 1 & 2 & 3 & 4 & 5 & 6  & 7  & 8  & 9  & 20  & 50  & 100 \\ \hline
			$p(n)$ & 1 & 1 & 2 & 3 & 5 & 7 & 11 & 15 & 22 & 30 & 627 & $\mathtt{\sim}$2e5 & $\mathtt{\sim}$2e8 \\
		\end{tabular}
		\end{center}

	\subsection{Binomials}
		\kactlimport{binomial.h}
		\kactlimport{binomialModPrime.h}
		\kactlimport{RollingBinomial.h}
		\kactlimport{multinomial.h}

	\subsection{Stirling numbers of the second kind}
		Partitions of $n$ distinct elements into exactly $k$ groups.
		$$S(n,k) = S(n-1,k-1) + k S(n-1,k)$$
		$$S(n,1) = S(n,n) = 1$$
		$$S(n,k) = \frac{1}{k!}\sum_{j=0}^k (-1)^{k-j}\binom{k}{j}j^n$$

	\subsection{Bell numbers}
		Total number of partitions of $n$ distinct elements.
		$$B(n) = \sum_{k=1}^n \binom{n-1}{k-1}B(n-k) = \sum_{k=1}^n S(n,k)$$
		$$B(0) = B(1) = 1$$
		The first are 1, 1, 2, 5, 15, 52, 203, 877, 4140, 21147, 115975, 678570, 4213597.
		For a prime $p$
		$$B(p^m+n)\equiv mB(n)+B(n+1) \pmod{p}$$

	\subsection{Triangles}
		Given rods of length $1,\ldots,n$,
		$$T(n) = \frac{1}{24} \left\{\begin{array}{ll}n(n-2)(2n-5) & n \text{ even}\\(n-1)(n-3)(2n-1) & n \text{ odd}\end{array}\right.$$
		is the number of distinct triangles (positive are) that can be constructed, i.e., the \# of 3-subsets of $[n]$ s.t. $x\leq y\leq z$ and $z\neq x+y$.

\section{General purpose numbers}
	\subsection{Catalan numbers}
		$$C_n=\frac{1}{n+1}\binom{2n}{n}= \binom{2n}{n}-\binom{2n}{n+1} = \frac{(2n)!}{(n+1)!n!}$$
		$$C_{n+1} = \frac{2(2n+1)}{n+2}C_n$$
		$$C_0=1, C_{n+1}=\sum C_iC_{n-i}$$
		First few are 1, 1, 2, 5, 14, 42, 132, 429, 1430, 4862, 16796, 58786, 208012, 742900.
		\begin{itemize}
			\setlength\itemsep{0pt}
			\item \# of monotonic lattice paths of a $n\times n$-grid which do not pass above the diagonal.
			\item \# of expressions containing $n$ pairs of parenthesis which are correctly matched.
			\item \# of full binary trees with with $n+1$ leaves (0 or 2 children).
			\item \# of non-isomorphic ordered trees with $n+1$ vertices.
			\item \# of ways a convex polygon with $n+2$ sides can be cut into triangles by connecting vertices with straight lines.
			\item \# of permutations of $[n]$ with no three-term increasing subsequence.
		\end{itemize}

	\subsection{Super Catalan numbers}
		The number of monotonic lattice paths of a $n\times n$-grid that do not touch the diagonal.
		$$S(n) = \frac{3(2n-3)S(n-1)-(n-3)S(n-2)}{n}$$
		$$S(1)=S(2)=1$$
		1, 1, 3, 11, 45, 197, 903, 4279, 20793, 103049, 518859

	\subsection{Motzkin numbers}
		Number of ways of drawing any number of nonintersecting chords among $n$ points on a circle. Number of lattice paths from $(0,0)$ to $(n,0)$ never going below the $x$-axis, using only steps NE, E, SE.
		$$M(n) = \frac{3(n-1)M(n-2)+(2n+1)M(n-1)}{n+2}$$
		$$M(0) = M(1) = 1$$
		1, 1, 2, 4, 9, 21, 51, 127, 323, 835, 2188, 5798, 15511, 41835, 113634

	\subsection{Narayana numbers}
		Number of lattice paths from $(0,0)$ to $(2n,0)$ never going below the $x$-axis, using only steps NE and SE, and with $k$ peaks.
		$$N(n,k) = \frac{1}{n}\binom{n}{k}\binom{n}{k-1}$$
		$$N(n,1) = N(n,n) = 1$$
		$$\sum_{k=1}^n N(n,k) = C_n$$
		1, 1, 1, 1, 3, 1, 1, 6, 6, 1, 1, 10, 20, 10, 1, 1, 15, 50

	\subsection{Schröder numbers}
		Number of lattice paths from $(0,0)$ to $(n,n)$ using only steps N,NE,E, never going above the diagonal. Number of lattice paths from $(0,0)$ to $(2n,0)$ using only steps NE, SE and double east EE, never going below the $x$-axis. Twice the Super Catalan number, except for the first term.
		1, 2, 6, 22, 90, 394, 1806, 8558, 41586, 206098

