% This is the LaTeX file for the chapter "Useful mathematical identities" in KACTL (KTH ACM Contest Template Library)
% Written by Håkan Terelius, 2009-02-10

\chapter{Useful mathematical identities}
%\section{Equations}
%\section{Trigonometry}
%\section{Geometry}
%\section{Derivatives/Integrals}

\section{Equations}
$$\begin{aligned}ax+by=c\\cx+dy=f\end{aligned}
\Rightarrow
\begin{aligned}x=\dfrac{ed-bf}{ad-bc}\\y=\dfrac{af-ec}{ad-bc}\end{aligned}$$
\section{Trigonometry}
\begin{align*}
\sin(v+w)&{}=\sin v\cos w+\cos v\sin w\\
cos(v+w)&{}=\cos v\cos w-\sin v\sin w\\
\tan(v+w)&{}=\dfrac{\tan v+\tan w}{1-\tan v\tan w}\\
\sin v+\sin w&{}=2\sin\dfrac{v+w}{2}\cos{v-w}{2}\\
\cos v+\cos w&{}=2\cos\dfrac{v+w}{2}\cos{v-w}{2}\\
(V+W)\tan(v-w)/2&{}=(V-W)\tan(v+w)/2
\end{align*}
where $V$ is the length of the side opposite the angle $v$, and
similar for $W$.
\begin{align*}
\begin{cases}
a\cos x+b\sin x=r\cos(x-\phi)\\
a\sin x+b\cos x=r\sin(x+\phi)
\end{cases}
\end{align*}
where $r=\sqrt{a^2+b^2}, \phi=\operatorname{atan2}(b,a)$.
\section{Spherical trigonometry}
$a,b,c=\text{sides}$, $\alpha,\beta,\gamma=\text{angles}$, all
six\ldots \emph{NOT DONE YET}
\begin{align*}
\cos a&{}=\cos b \cos c + \sin b \sin c \cos \alpha\\
\cos \alpha&{}=-\cos\beta \cos\gamma + \sin\beta\sin\gamma\cos a\\
\sin \alpha/\sin a&{}=\sin\beta\sin b=\sin\gamma/\sin c
\end{align*}
\section{Geometry}
\subsection{Triangles}
Side lengths $a,b,c$.\\
Semiperimeter $p=\dfrac{a+b+c}{2}$.\\
Area $A=\sqrt{p(p-a)(p-b)(p-c)}$.\\
Circumradius $R=\dfrac{abc}{4A}$.\\
Inradius $r=\dfrac{A}{p}$.\\
Median (divides triangle into two equal-sized triangles)
$m_a=\tfrac{1}{2}\sqrt{2b^2+2c^2-a^2}$.\\
Bisector (divides angles in two)
$s_a=\sqrt{bc\left[1-\left(\dfrac{a}{b+c}\right)^2\right]}$.
$$\frac{\sin\alpha}{a}=\frac{\sin\beta}{b}=\frac{\sin\gamma}{c}=\frac{1}{2R}$$
$$a^2=b^2+c^2-2bc\cos\alpha$$
$$\frac{a+b}{a-b}=\frac{\tan\dfrac{\alpha+\beta}{2}}{\tan\dfrac{\alpha-\beta}{2}}$$
\subsection{Quadrilaterals}
Side lengths $a,b,c,d$.\\
Diagonals $e(ad\leftrightarrow bc), f(ab\leftrightarrow cd)$.\\
Diagonals angle $\theta$.\\
Magic flux $F=b^2+d^2-a^2-c^2$.\\
Area $4A=2ef\sin\theta=F\tan\theta=\sqrt{4e^2f^2-F^2}$.
\section{Derivatives/Integrals}
\begin{align*}
\dfrac{d}{dx}\arcsin x&{}=\dfrac{1}{\sqrt{1-x^2}} &
\dfrac{d}{dx}\arccos x&{}=-\dfrac{1}{\sqrt{1-x^2}}\\
\dfrac{d}{dx}\tan x&{}=1+\tan^2 x &
\dfrac{d}{dx}\arctan x&{}=\dfrac{1}{1+x^2}\\
\int\tan ax&{}=-\dfrac{\ln|\cos ax|}{a} &
\int x\sin ax&{}=\dfrac{\sin ax-ax \cos ax}{a^2}
\end{align*}
