% This is the LaTeX file for the chapter "Various" in KACTL (KTH ACM Contest Template Library)
% Written by Håkan Terelius, 2009-02-10

\chapter{Various}

\section{Intervals}
	\kactlimport{intervalUnion.h}
	\kactlimport{intervalCover.h}
	\kactlimport{ConstantIntervals.h}

\section{Misc. algorithms}
	\kactlimport{TernarySearch.h}
	\kactlimport{Karatsuba.h}
	\kactlimport{lis.h}
	\kactlimport{lcs.h}

\section{Optimization tricks}
	\subsection{Bit hacks}
		\begin{itemize}
		\item \texttt{x \& -x} is the least bit in \texttt{x}.
		\item \texttt{for (int x = m; x; ) \{ --x \&= m; ... \}} loops over all subset masks of \texttt{m} (except \texttt{m} itself).
		\item \texttt{c = x\&-x, r = x+c; (((r\^{}x) >> 2)/c) | r} is the next number after \texttt{x} with the same number of bits set.
		\end{itemize}
	\subsection{Pragmas}
		\begin{itemize}
			\item \lstinline{#pragma GCC optimize ("Ofast")} will make GCC auto-vectorize for loops and optimizes floating points better.
			\item \lstinline{#pragma GCC target ("avx,avx2")} can double performance of vectorized code, but causes crashes on old machines.
		\end{itemize}
	\kactlimport{BumpAllocatorSTL.h}
	\kactlimport{BumpAllocator.h}
	\kactlimport{SIMD.h}
	\kactlimport{Unrolling.h}

\section{Hex and tri grids}
	\kactlimport{bricks.ps}
	\kactlimport{grids.ps}
