% This is the LaTeX file for the chapter "Various" in KACTL (KTH ACM Contest Template Library)
% Written by Håkan Terelius, 2009-02-10

\chapter{Various}

	\section{Intervals}
		\kactlimport{intervalUnion.h}
		\kactlimport{intervalCover.h}
		\kactlimport{ConstantIntervals.h}

	\section{Misc. algorithms}
		\kactlimport{TernarySearch.h}
		\kactlimport{lis.h}
		\kactlimport{lcs.h}
		\kactlimport{Karatsuba.h}

	\section{Optimization tricks}
		\subsection{Bit hacks}
			\begin{enumerate}
			\item \texttt{x \& -x} is the least bit in \texttt{x}.
			\item \texttt{for (int x = m; x; ) \{ --x \&= m; ... \}} loops over all subset masks of \texttt{m} (except \texttt{m} itself).
			\item \texttt{c = x\&-x, r = x+c; (((r\^{}x) >> 2)/c) | r} is the next number after \texttt{x} with the same number of bits set.
			\end{enumerate}
		\subsection{Pragmas}
			\lstinline{#pragma GCC target ("avx,avx2")} and \lstinline{#pragma GCC optimize ("Ofast")} may make GCC generate better code (especially for tight loops). Use when desperate.
		\kactlimport{SIMD.h}
		\kactlimport{Unrolling.h}

	\section{Hex and tri grids}
		\kactlimport{bricks.ps}
		\kactlimport{grids.ps}
